\section{Практическая часть}

\textbf{Приборы и материалы:} доска Гальтона, воронка, линейка, частицы - пшено, вольтметр В7-27, резисторы $R = 510\,$Ом $\pm10\%$, измеритель индуктивностей и ёмкостей высокочастотный Е7-5А, ёмкости $ C = 130\,$пФ $\pm5\%$, $C = 82\,$пФ $\pm10\%$

\subsection{Опыт с доской Гальтона}

Обозначим за $N_0$ - количество пшена в полном стакане, а $N$ - количества пшена в эксперименте. В опыте с $N = N_0 / 2$ и $N = N_0$, для удобства, будем измерять не количество пшена в ячейке, а высоту столбца в миллиметрах.
	
\subsubsection{Опыт с $N = 10$ зёрен}

Проведём эксперимент с доской Гальтона, используя $N = 10$ зёрен. Медленно сыпем пшено, и записываем сколько штук попало в каждую ячейку. Полученные данные представлены в виде таблицы (см. пункт \ref{pril_pract} Приложение к практической части, Таблица \ref{ap:table:1})

\subsubsection{Опыт с $N = N_0 / 2$ зёрен}

Проведём эксперимент с доской Гальтона, используя $N = N_0 / 2$ зёрен. Медленно сыпем пшено, и записываем сколько мм зёрен попало в каждую ячейку, внесём полученные данные в таблицу (см. пункт \ref{pril_pract} Приложение к практической части, Таблица \ref{ap:table:1})


\subsubsection{Опыт с $N = N_0$ зёрен} 

Проведём эксперимент с доской Гальтона, используя $N = N_0$ зёрен. Медленно сыпем пшено, и записываем сколько мм зёрен попало в каждую ячейку, внесём полученные данные в таблицу (см. пункт \ref{pril_pract} Приложение к практической части, Таблица \ref{ap:table:1})

\subsubsection{Обработка полученных данных}

Для удобства работы с полученными данными, представим их в графическом виде. Так в данном случае номер ячейки дискретная величина, то будем использовать закон для распределения случайной дискретной величины для доски Гальтона - формула \eqref{ver_formul}.
\subsection{Опыт с резисторами}

