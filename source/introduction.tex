\section{Введение}
\subsection{Цель}
Изучить колебательные движения на примере пружинного маятника.
\subsection{Задачи}
\begin{enumerate}
	\item Определение $k$ из формулы \ref{eqution:1}, измеряя для каждой пружины величины $\triangle l$ с различными грузами $M$.
	
	\item Измерить периоды колебаний для каждой пружины с различными грузами $M$. Построить графики зависимости $T^2$ от $M$. Сравнить с расчетной зависимостью $T^2 = \frac{4 \pi ^ 2 M}{k}$.
	
	\item Выяснить зависимость периода колебаний от амплитуды.
	
	\item Изучить зависимость периода колебаний от времени. Для чего нужно, на останавливая колебания, выяснить, через какое время амплитуда станет равной $\frac{3}{4}$ от начального значения $A_0$, затем $\frac{1}{2} A_0$, $\frac{1}{4} A_0$. Построить график зависимости A от t.
\end{enumerate}
\subsection{Приборы и оборудование}
Доска Гальтона, воронка, линейка, частицы - пшено, вольтметр В7-27, резисторы $R = 510\,$Ом $\pm10\%$, измеритель индуктивностей и ёмкостей высокочастотный Е7-5А, ёмкости $ C = 130\,$пФ $\pm5\%$, $C = 82\,$пФ $\pm10\%$