\section{Введение}
\subsection{Цель}

Изучить некоторые законы случайных событий, рассмотреть примеры дискретной и непрерывно	величин. 
\subsection{Задачи}
\begin{enumerate}
	\item Провести эксперимент с доской Гальтона.
	
	\item Сравнить полученные результаты с теорией для распределения случайной дискретной величины. 
	
	\item Провести эксперимент с измерением 100 резисторов.
	
	\item  Сравнить полученные результаты с теорией для распределения случайной непрерывной величины.
	
\end{enumerate}
\subsection{Приборы и оборудование}
Доска Гальтона, воронка, линейка, частицы - пшено, вольтметр В7-27, резисторы $R = 510\,$Ом $\pm10\%$, измеритель индуктивностей и ёмкостей высокочастотный Е7-5А, ёмкости $ C = 130\,$пФ $\pm5\%$, $C = 82\,$пФ $\pm10\%$