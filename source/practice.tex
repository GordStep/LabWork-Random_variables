\section{Практическая часть}



\subsection{Опыт с доской Гальтона}

$\qquad$ Обозначим за $N_0$ - количество пшена в полном стакане, а $N$ - количества пшена в эксперименте. В опыте с $N = N_0 / 2$ и $N = N_0$, для удобства, будем измерять не количество пшена в ячейке, а высоту столбца в миллиметрах.
	
\subsubsection{Опыт с $N = 10$ зёрен}

$\qquad$ Проведём эксперимент с доской Гальтона, используя $N = 10$ зёрен. Медленно сыпем пшено, и записываем сколько штук попало в каждую ячейку. Полученные данные представлены в виде таблицы (см. пункт \ref{pril_practic_table}, Таблица \ref{ap:table:1})

\subsubsection{Опыт с $N = N_0 / 2$ зёрен}

$\qquad$ Проведём эксперимент с доской Гальтона, используя $N = N_0 / 2$ зёрен. Медленно сыпем пшено, и записываем сколько мм зёрен попало в каждую ячейку, внесём полученные данные в таблицу (см. пункт \ref{pril_practic_table}, Таблица \ref{ap:table:1})


\subsubsection{Опыт с $N = N_0$ зёрен} 

$\qquad$ Проведём эксперимент с доской Гальтона, используя $N = N_0$ зёрен. Медленно сыпем пшено, и записываем сколько мм зёрен попало в каждую ячейку, внесём полученные данные в таблицу (см. пункт \ref{pril_practic_table}, Таблица \ref{ap:table:1})

\subsubsection{Обработка полученных данных} \label{obrabotka}

$\qquad$ Для удобства работы с данными, представим результаты экспериментов в графическом виде (см. рис. \ref{fig:graph-1} и рис. \ref{fig:graph-2}). Так как в данном случае номер ячейки дискретная величина, то для поиска теоретической вероятности будем использовать закон для распределения случайной дискретной величины, в случае доски Гальтона он имеет вид \eqref{ver_formul}.

Для построения теоретической функции, найдём максимальную вероятность в эксперименте. Для доски Гальтона она равна $P(\overline{k})$ (вероятности попадания в среднюю точку). В нашем эксперименте $P(\overline{k})_{N_0/2} = 0,0527 \frac{1}{\text{мм}}$ и $P(\overline{k})_{N_0} = 0,0528 \frac{1}{\text{мм}}$. Дальше найдём $\sigma_k$, выразив её из формулы \eqref{th:3:1}. Получим:

\begin{align} \label{prac:1}
	\sigma_k = \frac{1}{\sqrt{2 \pi} P(\overline{k})} 
\end{align}

Найдём по формуле \eqref{prac:1} ${\sigma_k}_{N_0/2}$ и ${\sigma_k}_{N_0}$ 

\begin{align*}
	{\sigma_k}_{N_0/2} = \frac{1}{\sqrt{2 \pi} P(\overline{k})} = \frac{1}{\sqrt{2 \pi} \times 0,0527 \frac{1}{\text{мм}}} \approx 7,57 \, \text{мм}
\end{align*}

\begin{align*}
	{\sigma_k}_{N_0} = \frac{1}{\sqrt{2 \pi} P(\overline{k})} = \frac{1}{\sqrt{2 \pi} \times 0,0527 \frac{1}{\text{мм}}} \approx 7,56 \, \text{мм}
\end{align*}

Так как в нашей установке 54 столбца, пронумерованные от 1 до 54, то за среднюю ячейку с самой большой вероятностью примем точку $k = 27.5$ 

С помощью программы на языке программирования $Python$ найдём теоретическую кривую вероятности по формуле \eqref{th:3:1}, на графиках (см. рис. \ref{fig:graph-1} и рис. \ref{fig:graph-2}) отметим её серым пунктиром.

\subsubsection{Сравнение флуктуаций}

\begin{enumerate}
	\item 
	{
		Для $N = N_0 / 2$ максимальная вероятность $P(k) \approx 0,53 \, \frac{1}{\text{мм}}$ соответствует $\overline{k} = 28$, найдём относительную флуктуацию $\eta = {\sigma_k}/{\overline{k}} = \frac{7,56 \, \text{мм}}{28} \approx 0,27 \, \text{мм}$
	}
	\item 
	{
		Для $N = N_0$ максимальная вероятность $P(k) \approx 0,53 \, \frac{1}{\text{мм}}$ соответствует $\overline{k} = 28$, найдём относительную флуктуацию $\eta = {\sigma_k}/{\overline{k}} = \frac{7,57 \, \text{мм}}{28} \approx 0,27 \, \text{мм}$
	}
\end{enumerate}


\subsubsection{Сравнение стандартов}

Из пункта \ref{obrabotka} мы получили ${\sigma_k}_{N_0 / 2} \approx 7,56 \, \text{мм}$ и ${\sigma_k}_{N_0} \approx 7,57 \, \text{мм}$. Найдём значение стандарта графически, как полуширину экспериментальной кривой: 

\begin{enumerate}
	\item 
	{
		Для $N = N_0 / 2$ (см. пункт \ref{pril_practic_graph}, рис. \ref{fig:graph-1}): $\sigma_{\text{эксп}} \approx 7,55 \, \text{мм}$. 
	}
	\item 
	{
		Для $N = N_0$ (см. пункт \ref{pril_practic_graph}, рис. \ref{fig:graph-2}): $\sigma_{\text{эксп}} \approx 7,55 \, \text{мм}$
	}
\end{enumerate}

С учётом погрешности прямого измерения в $0,5 \, \text{мм}$, теоретический и экспериментальный стандарты совпадают. Следовательно выполняется закон Гаусса.

\subsubsection{Вывод по эксперименту с доской Гальтона}

По эксперименту с $N = 10$, мы не можем выделить закономерности попадания пшена в ячейки, следовательно выборка слишком мала. 

Сравнение графиков (см. пункт \ref{pril_practic_graph}, рис. \ref{fig:graph-1} и рис. \ref{fig:graph-2}) показывает, что выборка $N = N_0/2$ примерно равна выборке $N = N_0$, и они совпадают с математической функцией распределения. Следовательно выборка $N = N_0/2$ является достаточной для оценки плотности вероятности. Сравнив теоретический и экспериментальный стандарты $\sigma_k$, мы показали, что выполняется закон Гаусса.

\subsection{Опыт с резисторами}
 
Обозначим за $N$ количество резисторов. Для эксперимента измерим сопротивление $N = 100$ резисторов сопротивлением $R = 510\,\text{Ом}\pm10\%$. 

\subsubsection{Обработка полученных результатов}
Полученные данные внесём в таблицу (см. пункт \ref{pril_practic_table}, Таблица \ref{ap:table:2}). Для удобства работы с данными и построения графиков представим данные в виде(см. пункт \ref{pril_practic_table}, Таблица \ref{ap:table:3}).

Построим функцию $F(r)$ - приближённое представление интегральной функции распределения, с помощью приблизительного представления (см. пункт \ref{pril_practic_graph}, рис. \ref{fig:graph-3})

\begin{align}
	F(r) = P(R < r) \approx \frac{N'}{N},
\end{align} 
где $N'$ - число значений $R$, которые меньше $r$.

Путем графического дифференцирования функции $F(r)$ построим плотность вероятностей $W(r)$(см. пункт \ref{pril_practic_graph}, рис. \ref{fig:graph-5}). На том же графике построим нормальное распределение

\begin{align}
	W_T(r) = \frac{1}{\sqrt{2 \pi} \sigma} \exp \left(- \frac{(r - \overline{R})^2}{2 \sigma^2}\right)
\end{align}

Где в качестве $\overline{R}$ мы возьмём наиболее вероятное значение $r$. Из наших данных это будет $\overline{R} = 510\, \text{Ом}$, с вероятностью $P(510) = 0,11 \, \frac{1}{\text{Ом}}$. За $\sigma$ мы возьмём полуширину экспериментальной кривой $W(r)$ на уровне 
\begin{align*}
		P(x) = \frac{W(\overline{R})}{\sqrt{e}} = \frac{0.11}{\sqrt{e}} \approx 0,067\, \frac{1}{\text{Ом}}
\end{align*}

Найдём эту полуширину графически с помощью вспомогательных линий (см. пункт \ref{pril_practic_graph}, рис. \ref{fig:graph-4}). Получаем значение $\sigma = 3,59 \, \text{Ом}$. Построим итоговый график $W_T(r)$ (см. пункт \ref{pril_practic_graph}, рис. \ref{fig:graph-5}). 

\subsection{Оценка погрешностей}

\subsubsection{Стандарт $\sigma$}

Сравним стандарт $\sigma$ с погрешностью мостика Уитстона, использованного для измерений. 

Найдём стандарт для нашего измерения по формуле \eqref{13}, получим $D_k \approx 95,2 \, \text{мм}$, а
\begin{align*}
	\sigma_{\text{моста}} = \sqrt{D_k} \approx 9,76 \, \text{мм}
\end{align*}

Найдём отношение $\sigma$ и $\sigma_{\text{моста}}$:
\begin{align*}
	\frac{\sigma_{\text{моста}}}{\sigma} = \frac{9,76}{3,59} \approx 2,72
\end{align*}

Следовательно оценка с помощью $\sigma$, найденного графическим способом точнее, чем стандарт прямого измерения.

\subsubsection{Номинальное значение}

Сравним $\overline{R}$ с номинальным значением сопротивления $R_0$, в нашем случае они равны
\begin{align*}
	\overline{R} = R_0 = 510 \, \text{Ом}
\end{align*}

Следовательно систематическая погрешность $1 - R_0 / \overline{R} = 0$.
Случайную погрешности изготовления сравним с погрешностью изготовления, указанной на резисторах.
Случайная погрешность равна $\sigma / \overline{R} = 3,59 \, \text{Ом} / 510 \, \text{Ом} \approx 0,01$. На резисторах указана погрешность $10\%$, т.е. $0,1$. В нашем случае получилось, что случайная погрешность меньше погрешности заводской погрешности.
