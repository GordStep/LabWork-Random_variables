\section{Практическая часть}



\subsection{Опыт с доской Гальтона}

$\qquad$ Обозначим за $N_0$ - количество пшена в полном стакане, а $N$ - количества пшена в эксперименте. В опыте с $N = N_0 / 2$ и $N = N_0$, для удобства, будем измерять не количество пшена в ячейке, а высоту столбца в миллиметрах.
	
\subsubsection{Опыт с $N = 10$ зёрен}

$\qquad$ Проведём эксперимент с доской Гальтона, используя $N = 10$ зёрен. Медленно сыпем пшено, и записываем сколько штук попало в каждую ячейку. Полученные данные представлены в виде таблицы (см. пункт \ref{pril_practic_table}, Таблица \ref{ap:table:1})

\subsubsection{Опыт с $N = N_0 / 2$ зёрен}

$\qquad$ Проведём эксперимент с доской Гальтона, используя $N = N_0 / 2$ зёрен. Медленно сыпем пшено, и записываем сколько мм зёрен попало в каждую ячейку, внесём полученные данные в таблицу (см. пункт \ref{pril_practic_table}, Таблица \ref{ap:table:1})


\subsubsection{Опыт с $N = N_0$ зёрен} 

$\qquad$ Проведём эксперимент с доской Гальтона, используя $N = N_0$ зёрен. Медленно сыпем пшено, и записываем сколько мм зёрен попало в каждую ячейку, внесём полученные данные в таблицу (см. пункт \ref{pril_practic_table}, Таблица \ref{ap:table:1})

\subsubsection{Обработка полученных данных} \label{obrabotka}
Погрешность прямого измерения $\Delta h = \pm 1 \text{ мм}$, после деления на сумму высот столбцов, которая равна 1444,2 мм, мы получим погрешность вероятности $\approx 0,0007 \text{ мм}$, этой величиной можно пренебречь.

$\qquad$ Для удобства работы с данными, представим результаты экспериментов в графическом виде (см. рис. \ref{fig:graph-1} и рис. \ref{fig:graph-2}). Так как в данном случае номер ячейки дискретная величина, то для поиска теоретической вероятности будем использовать закон для распределения случайной дискретной величины, в случае доски Гальтона он имеет вид \eqref{ver_formul}.

Для построения теоретической функции, найдём максимальную вероятность в эксперименте. Для доски Гальтона она равна $P(\overline{k})$ (вероятности попадания в среднюю точку). В нашем эксперименте $P(\overline{k})_{N_0/2} = 0,0527 \frac{1}{\text{мм}}$ и $P(\overline{k})_{N_0} = 0,0528 \frac{1}{\text{мм}}$. Дальше найдём $\sigma_k$, выразив её из формулы \eqref{th:3:1}. Получим:

\begin{align} \label{prac:1}
	\sigma_k = \frac{1}{\sqrt{2 \pi} P(\overline{k})} 
\end{align}

Найдём по формуле \eqref{prac:1} ${\sigma_k}_{N_0/2}$ и ${\sigma_k}_{N_0}$ 

\begin{align*}
	{\sigma_k}_{N_0/2} = \frac{1}{\sqrt{2 \pi} P(\overline{k})} = \frac{1}{\sqrt{2 \pi} \times 0,0527 \frac{1}{\text{мм}}} \approx 7,570 \, \text{мм}
\end{align*}

\begin{align*}
	{\sigma_k}_{N_0} = \frac{1}{\sqrt{2 \pi} P(\overline{k})} = \frac{1}{\sqrt{2 \pi} \times 0,0527 \frac{1}{\text{мм}}} \approx 7,570 \, \text{мм}
\end{align*}

Так как в нашей установке 54 столбца, пронумерованные от 1 до 54, то за среднюю ячейку с самой большой вероятностью примем точку $k = 27.5$ 

С помощью программы на языке программирования $Python$ найдём теоретическую кривую вероятности по формуле \eqref{th:3:1}, на графиках (см. рис. \ref{fig:graph-1} и рис. \ref{fig:graph-2}) отметим её серым пунктиром и аппроксимируем практические данные полиномом 10-ой степени и отменим его серым пунктиром на графике (см. рис. \ref{fig:graph-1} и рис. \ref{fig:graph-2}).

 

\subsubsection{Сравнение флуктуаций в средней ячейке}

\begin{enumerate}
	\item 
	{
		Для $N = N_0 / 2$ максимальная вероятность $P(k) \approx 0,053 \, \frac{1}{\text{мм}}$ соответствует $\overline{k} = 28$, найдём относительную флуктуацию $\eta = \sqrt{\frac{1 - p}{N p}} = \sqrt{\frac{1 - 0,053}{1445,2 \cdot 0,053}} \approx 0,111 \, \text{мм}$
	}
	\item 
	{
		Для $N = N_0$ максимальная вероятность $P(k) \approx 0,053 \, \frac{1}{\text{мм}}$ соответствует $\overline{k} = 28$, найдём относительную флуктуацию $\eta = \sqrt{\frac{1 - p}{N p}} = \sqrt{\frac{1 - 0,051}{1445,2 \cdot 0,051}} \approx 0,096 \, \text{мм}$
	}
\end{enumerate}

Найдём отношение флуктуаций 
\[\frac{\eta_{N_0/2}}{\eta_{N_0}} \approx \frac{0,11}{0,096} \approx 1,146\]
Следовательно при увеличении количества испытаний флуктуация уменьшится.

\subsubsection{Сравнение флуктуаций в средней и крайней ячейках}

Из предыдущего пункта возьмём флуктуацию для $N = N_0 / 2 = 1445,2 \text{ мм}, \eta \approx 0,270 \, \text{мм}$.

Рассчитаем флуктуацию для ячейки $k = 6$, для неё $ p = P(6) \approx 0,0014$ для нахождения $\eta_{k = 6}$ воспользуемся формулой \eqref{16}:
\[ \eta_{k = 6} = \sqrt{\frac{1 - p}{N p}} = \sqrt{\frac{1 - 0,0014} {1445,5 \cdot 0,0014}} \approx 0,710 \]

Сравним флуктуации в выбранных точках:
\[ \frac{\eta_{k = 6}}{\eta_{\overline{k}}} = \frac{0,71}{0,11} \approx 6,45 \]
Следовательно, флуктуация в крайней ячейке в 6,45 раз больше, чем флуктуация в средней ячейке.

\subsubsection{Сравнение стандартов}

Из пункта \ref{obrabotka} мы получили ${\sigma_k}_{N_0 / 2} \approx 7,570 \, \text{мм}$ и ${\sigma_k}_{N_0} \approx 7,57 \, \text{мм}$. Найдём значение стандарта графически, как полуширину экспериментальной кривой: 

\begin{enumerate}
	\item 
	{
		Для $N = N_0 / 2$ (см. пункт \ref{pril_practic_graph}, рис. \ref{fig:graph-1}): $\sigma_{\text{эксп}} \approx 7,55 \, \text{мм}$. 
	}
	\item 
	{
		Для $N = N_0$ (см. пункт \ref{pril_practic_graph}, рис. \ref{fig:graph-2}): $\sigma_{\text{эксп}} \approx 7,55 \, \text{мм}$
	}
\end{enumerate}

С учётом погрешности прямого измерения в $0,5 \, \text{мм}$, теоретический и экспериментальный стандарты совпадают. Следовательно выполняется закон Гаусса.

\subsubsection{Вывод по эксперименту с доской Гальтона}

По эксперименту с $N = 10$, мы не можем выделить закономерности попадания пшена в ячейки, следовательно выборка слишком мала. 

Сравнение графиков (см. пункт \ref{pril_practic_graph}, рис. \ref{fig:graph-1} и рис. \ref{fig:graph-2}) показывает, что выборка $N = N_0/2$ примерно равна выборке $N = N_0$, и они совпадают с математической функцией распределения. Следовательно выборка $N = N_0/2$ является достаточной для оценки плотности вероятности. Сравнив теоретический и экспериментальный стандарты $\sigma_k$, мы показали, что выполняется закон Гаусса.

\subsection{Опыт с резисторами}
 
Обозначим за $N$ количество резисторов. Для эксперимента измерим сопротивление $N = 100$ резисторов сопротивлением $R = 510\,\text{Ом}\pm10\%$. 

\subsubsection{Обработка полученных результатов}
Полученные данные внесём в таблицу (см. пункт \ref{pril_practic_table}, Таблица \ref{ap:table:2}). Для удобства работы с данными и построения графиков представим данные в виде(см. пункт \ref{pril_practic_table}, Таблица \ref{ap:table:3}).

Построим функцию $F(r)$ - приближённое представление интегральной функции распределения, с помощью приблизительного представления (см. пункт \ref{pril_practic_graph}, рис. \ref{fig:graph-3})

\begin{align}
	F(r) = P(R < r) \approx \frac{N'}{N},
\end{align} 
где $N'$ - число значений $R$, которые меньше $r$.

Для удобства аппроксимируем полученные данные с помощью сигмойды
\begin{align*}
	y = \frac{a}{1 + \exp(-k (x - x_0))} + c
\end{align*}

С коэффициентами $ a = 1, k = 0.23, x_0 = 511, c = 0 $

Путем графического дифференцирования аппроксимации дискретной функции $F(r)$ построим плотность вероятностей $W(r)$(см. пункт \ref{pril_practic_graph}, рис. \ref{fig:graph-4}). На том же графике построим нормальное распределение

\begin{align}
	W_T(r) = \frac{1}{\sqrt{2 \pi} \sigma} \exp \left(- \frac{(r - \overline{R})^2}{2 \sigma^2}\right)
\end{align}

Где в качестве $\overline{R}$ мы возьмём наиболее вероятное значение $r$. Из наших данных это будет $\overline{R} = 511,2\, \text{Ом}$, с вероятностью $P(511,2) = 0,057 \, \frac{1}{\text{Ом}}$. За $\sigma$ мы возьмём полуширину экспериментальной кривой $W(r)$ на уровне 
\begin{align*}
		P(x) = \frac{W(\overline{R})}{\sqrt{e}} = \frac{0,057}{\sqrt{e}} \approx 0,021\, \frac{1}{\text{Ом}}
\end{align*}

Найдём эту полуширину графически с помощью вспомогательных линий (см. пункт \ref{pril_practic_graph}, рис. \ref{fig:graph-4}). Получаем значение $\sigma = 6,37 \, \text{Ом}$. Построим итоговый график $W_T(r)$ (см. пункт \ref{pril_practic_graph}, рис. \ref{fig:graph-4}). 

\subsection{Оценка погрешностей опыта с резисторами}

Погрешность прямого измерения $\Delta R = \pm 1 \text{ Ом}$.

\subsubsection{Стандарт $\sigma$}

Сравним стандарт $\sigma$ с погрешностью мостика Уитстона, использованного для измерений. 

Найдём стандарт для нашего измерения по формуле \eqref{13}, получим $D_k \approx 95,2 \, \text{мм}$, а
\begin{align*}
	\sigma_{\text{моста}} = \sqrt{D_k} \approx 9,76 \, \text{мм}
\end{align*}

Найдём отношение $\sigma$ и $\sigma_{\text{моста}}$:
\begin{align*}
	\frac{\sigma_{\text{моста}}}{\sigma} = \frac{9,76}{6,37} \approx 1,53
\end{align*}

Следовательно оценка с помощью $\sigma$, найденного графическим способом точнее, чем стандарт прямого измерения.

\subsubsection{Номинальное значение}

Сравним $\overline{R} = 511,2 \text{ Ом}$ с номинальным значением сопротивления $R_0 = 510 \text{ Ом}$, в нашем случае они равны
\begin{align*}
	\overline{R} - R_0 = 511,2 - 510 \text{ Ом} = 1,2 \text{ Ом} > \Delta R
\end{align*}
Следовательно, смещение среднего значения больше погрешности, но оно укладывается в погрешность изготовления.

Следовательно, систематическая погрешность $1 - R_0 / \overline{R} = 1 - \frac{510}{511,2} \approx 0,002$.

А случайная погрешность $\sigma / \overline{R} = \frac{6,37 \text{ Ом}}{511,2 \text{ Ом}} \approx 0,012 $

Сравним случайную и систематическую погрешности с погрешность изготовления, указанной на резисторах. Она больше: $ 10\% = 0,1 > 0,012 > 0,002 $. Следовательно эти погрешности не выходят из диапазона допустимых значений сопротивления резисторов.
