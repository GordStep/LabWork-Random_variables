\section{Практическая часть}



\subsection{Опыт с доской Гальтона}

$\qquad$ Обозначим за $N_0$ - количество пшена в полном стакане, а $N$ - количества пшена в эксперименте. В опыте с $N = N_0 / 2$ и $N = N_0$, для удобства, будем измерять не количество пшена в ячейке, а высоту столбца в миллиметрах.
	
\subsubsection{Опыт с $N = 10$ зёрен}

$\qquad$ Проведём эксперимент с доской Гальтона, используя $N = 10$ зёрен. Медленно сыпем пшено, и записываем сколько штук попало в каждую ячейку. Полученные данные представлены в виде таблицы (см. пункт \ref{pril_practic_table}, Таблица \ref{ap:table:1})

\subsubsection{Опыт с $N = N_0 / 2$ зёрен}

$\qquad$ Проведём эксперимент с доской Гальтона, используя $N = N_0 / 2$ зёрен. Медленно сыпем пшено, и записываем сколько мм зёрен попало в каждую ячейку, внесём полученные данные в таблицу (см. пункт \ref{pril_practic_table}, Таблица \ref{ap:table:1})


\subsubsection{Опыт с $N = N_0$ зёрен} 

$\qquad$ Проведём эксперимент с доской Гальтона, используя $N = N_0$ зёрен. Медленно сыпем пшено, и записываем сколько мм зёрен попало в каждую ячейку, внесём полученные данные в таблицу (см. пункт \ref{pril_practic_table}, Таблица \ref{ap:table:1})

\subsubsection{Обработка полученных данных}

$\qquad$ Для удобства работы с данными, представим результаты экспериментов в графическом виде (см. рис. \ref{fig:graph-1} и рис. \ref{fig:graph-2}). Так как в данном случае номер ячейки дискретная величина, то для поиска теоретической вероятности будем использовать закон для распределения случайной дискретной величины, в случае доски Гальтона он имеет вид \eqref{ver_formul}.

Для построения теоретической функции, найдём максимальную вероятность в эксперименте. Для доски Гальтона она равна $P(\overline{k})$ (вероятности попадания в среднюю точку). В нашем эксперименте $P(\overline{k})_{N_0/2} = 0,0527 \frac{1}{\text{мм}}$ и $P(\overline{k})_{N_0} = 0,0528 \frac{1}{\text{мм}}$. Дальше найдём $\sigma_k$, выразив её из формулы \eqref{th:3:1}. Получим:

\begin{align} \label{prac:1}
	\sigma_k = \frac{1}{\sqrt{2 \pi} P(\overline{k})} 
\end{align}

Найдём по формуле \eqref{prac:1} ${\sigma_k}_{N_0/2}$ и ${\sigma_k}_{N_0}$ 

\begin{align*}
	{\sigma_k}_{N_0/2} = \frac{1}{\sqrt{2 \pi} P(\overline{k})} = \frac{1}{\sqrt{2 \pi} \times 0,0527 \frac{1}{\text{мм}}} \approx 7,57 \, \text{мм}
\end{align*}

\begin{align*}
	{\sigma_k}_{N_0} = \frac{1}{\sqrt{2 \pi} P(\overline{k})} = \frac{1}{\sqrt{2 \pi} \times 0,0527 \frac{1}{\text{мм}}} \approx 7,56 \, \text{мм}
\end{align*}

Так как в нашей установке 54 столбца, пронумерованные от 1 до 54, то за среднюю ячейку с самой большой вероятностью примем точку $k = 27.5$ 

С помощью программы на языке программирования $Python$ найдём теоретическую кривую вероятности по формуле \eqref{th:3:1}, на графиках (см. рис. \ref{fig:graph-1} и рис. \ref{fig:graph-2}) отметим её серым пунктиром.

\subsubsection{Вывод по эксперименту с доской Гальтона}

По эксперименту с $N = 10$, мы не можем выделить закономерности попадания пшена в ячейки, следовательно выборка слишком мала. 

Сравнение графиков (см. рис. \ref{fig:graph-1} и рис. \ref{fig:graph-2}) показывает, что выборка $N = N_0/2$ примерно равна выборке $N = N_0$, и они совпадают с математической функцией распределения. Следовательно выборка $N = N_0/2$ является достаточной для оценки плотности вероятности.

\subsection{Опыт с резисторами}
 
Обозначим за $N$ количество резисторов. Для эксперимента измерим сопротивление $N = 100$ резисторов сопротивлением $R = 510\,\text{Ом}\pm10\%$. 

\subsubsection{Обработка полученных результатов}
Полученные данные внесём в таблицу (см. пункт \ref{pril_practic_table}, Таблица \ref{ap:table:2}). Для удобства работы с данными и построения графиков представим данные в виде(см. пункт \ref{pril_practic_table}, Таблица \ref{ap:table:3}).

Построим функцию $F(r)$ - приближённое представление интегральной функции распределения, с помощью приблизительного представления 

\begin{align}
	F(r) = P(R < r) \approx \frac{N'}{N}
\end{align}
, где $N'$ - число значений $R$, которые меньше $r$.
